% Some basic packages
\documentclass[letterpaper]{article}
\usepackage[utf8]{inputenc}
\usepackage[T1]{fontenc}
\usepackage{textcomp}
\usepackage[english]{babel}
\usepackage{url}
\usepackage{graphicx}
\usepackage{float}
\usepackage{booktabs}
\usepackage{enumitem}
\usepackage{adjustbox}
\usepackage{hyperref}
\usepackage[binary-units]{siunitx}
\sisetup{
	round-mode=places,
	round-precision=4
}



%margins
\usepackage[margin=0.75in]{geometry}

% Don't indent paragraphs, leave some space between them
\usepackage{parskip}

% Hide page number when page is empty
\usepackage{emptypage}
\usepackage{subcaption}
\usepackage{multicol}
\usepackage{xcolor}

% Other font I sometimes use.
% \usepackage{cmbright}

% Math stuff
\usepackage{amsmath, amsfonts, mathtools, amsthm, amssymb}
% Fancy script capitals
\usepackage{mathrsfs}
\usepackage{cancel}
% Bold math
\usepackage{bm}
% Some shortcuts
\newcommand\N{\ensuremath{\mathbb{N}}}
\newcommand\R{\ensuremath{\mathbb{R}}}
\newcommand\Z{\ensuremath{\mathbb{Z}}}
\renewcommand\O{\ensuremath{\emptyset}}
\newcommand\Q{\ensuremath{\mathbb{Q}}}
\newcommand\C{\ensuremath{\mathbb{C}}}

% Easily typeset systems of equations (French package)
\usepackage{systeme}

% Put x \to \infty below \lim
\let\svlim\lim\def\lim{\svlim\limits}

%Make implies and impliedby shorter
\let\implies\Rightarrow
\let\impliedby\Leftarrow
\let\iff\Leftrightarrow
\let\epsilon\varepsilon

% Add \contra symbol to denote contradiction
\usepackage{stmaryrd} % for \lightning
\newcommand\contra{\scalebox{1.5}{$\lightning$}}

% \let\phi\varphi

% Command for short corrections
% Usage: 1+1=\correct{3}{2}

\definecolor{correct}{HTML}{009900}
\newcommand\correct[2]{\ensuremath{\:}{\color{red}{#1}}\ensuremath{\to }{\color{correct}{#2}}\ensuremath{\:}}
\newcommand\green[1]{{\color{correct}{#1}}}

% horizontal rule
\newcommand\hr{
    \noindent\rule[0.5ex]{\linewidth}{0.5pt}
}

% hide parts
\newcommand\hide[1]{}

% si unitx
\usepackage{siunitx}
\sisetup{locale = US}

% Environments
\makeatother
% For box around Definition, Theorem, \ldots
\usepackage{mdframed}
\mdfsetup{skipabove=1em,skipbelow=0em}
\theoremstyle{definition}
\newmdtheoremenv[nobreak=true]{lemma}{Lemma}
\newmdtheoremenv[nobreak=true]{postulate}{Postulate}
\newmdtheoremenv{conclusion}{Conclusion}
\newmdtheoremenv{assume}{Assume}
\newtheorem*{observe}{Observe}
\newtheorem*{comment}{Comment}
\newtheorem*{practice}{Practice}
\newtheorem*{problem}{Problem}
\newtheorem*{terminology}{Terminology}
\newtheorem*{application}{Application}
\newtheorem*{question}{Question}

\newmdtheoremenv[nobreak=true]{definition}{Definition}
\newtheorem*{eg}{Example}
\newtheorem*{notation}{Notation}
\newtheorem*{previouslyseen}{As previously seen}
\newtheorem*{remark}{Remark}
\newtheorem*{note}{Note}
\newtheorem*{property}{Property}
\newtheorem*{intuition}{Intuition}
\newmdtheoremenv[nobreak=true]{prop}{Proposition}
\newmdtheoremenv[nobreak=true]{theorem}{Theorem}
\newmdtheoremenv[nobreak=true]{corollary}{Corollary}

% End example and intermezzo environments with a small diamond (just like proof
% environments end with a small square)
\usepackage{etoolbox}
\AtEndEnvironment{vb}{\null\hfill$\diamond$}%
\AtEndEnvironment{intermezzo}{\null\hfill$\diamond$}%
% \AtEndEnvironment{opmerking}{\null\hfill$\diamond$}%

% Fix some spacing
% http://tex.stackexchange.com/questions/22119/how-can-i-change-the-spacing-before-theorems-with-amsthm
\makeatletter
\def\thm@space@setup{%
  \thm@preskip=\parskip \thm@postskip=0pt
}


% Exercise 
% Usage:
% \exercise{5}
% \subexercise{1}
% \subexercise{2}
% \subexercise{3}
% gives
% Exercise 5
%   Exercise 5.1
%   Exercise 5.2
%   Exercise 5.3
\newcommand{\exercise}[1]{%
    \def\@exercise{#1}%
    \subsection*{Exercise #1}
}

\newcommand{\subexercise}[1]{%
    \subsubsection*{Exercise \@exercise.#1}
}


% \lecture starts a new lecture (les in dutch)
%
% Usage:
% \lecture{1}{di 12 feb 2019 16:00}{Inleiding}
%
% This adds a section heading with the number / title of the lecture and a
% margin paragraph with the date.

% I use \dateparts here to hide the year (2019). This way, I can easily parse
% the date of each lecture unambiguously while still having a human-friendly
% short format printed to the pdf.

\usepackage{xifthen}
\def\testdateparts#1{\dateparts#1\relax}
\def\dateparts#1 #2 #3 #4 #5\relax{
    \marginpar{\small\textsf{\mbox{#1 #2 #3 #5}}}
}

% \def\@lecture{}%
% \newcommand{\lecture}[3]{
%     \ifthenelse{\isempty{#3}}{%
%         \def\@lecture{Lecture #1}%
%     }{%
%         \def\@lecture{Lecture #1: #3}%
%     }%
%     \subsection*{\@lecture}
%     \marginpar{\small\textsf{\mbox{#2}}}
% }



% These are the fancy headers
\usepackage{fancyhdr}
\pagestyle{fancy}

% LE: left even
% RO: right odd
% CE, CO: center even, center odd
% My name for when I print my lecture notes to use for an open book exam.
% \fancyhead[LE,RO]{Jared Anderson}

% \fancyhead[RO,LE]{\@lecture} % Right odd,  Left even
\fancyhead[RE,LO]{}          % Right even, Left odd

\fancyfoot[RO,LE]{\thepage}  % Right odd,  Left even
\fancyfoot[RE,LO]{}          % Right even, Left odd
\fancyfoot[C]{\leftmark}     % Center

\makeatother




% Todonotes and inline notes in fancy boxes
\usepackage{todonotes}
\usepackage{tcolorbox}

% Make boxes breakable
\tcbuselibrary{breakable}

% Verbetering is correction in Dutch
% Usage: 
% \begin{verbetering}
%     Lorem ipsum dolor sit amet, consetetur sadipscing elitr, sed diam nonumy eirmod
%     tempor invidunt ut labore et dolore magna aliquyam erat, sed diam voluptua. At
%     vero eos et accusam et justo duo dolores et ea rebum. Stet clita kasd gubergren,
%     no sea takimata sanctus est Lorem ipsum dolor sit amet.
% \end{verbetering}
\newenvironment{correction}{\begin{tcolorbox}[
    arc=0mm,
    colback=white,
    colframe=green!60!black,
    title=Opmerking,
    fonttitle=\sffamily,
    breakable
]}{\end{tcolorbox}}

% Noot is note in Dutch. Same as 'verbetering' but color of box is different
\newenvironment{aside}[1]{\begin{tcolorbox}[
    arc=0mm,
    colback=white,
    colframe=white!60!black,
    title=#1,
    fonttitle=\sffamily,
    breakable
]}{\end{tcolorbox}}




% Figure support 
\usepackage{import}
\usepackage{xifthen}
% \pdfminorversion=7
\usepackage{pdfpages}
\usepackage{transparent}
\newcommand{\incfig}[2][1]{%
    \def\svgwidth{#1\columnwidth}
    \import{./figures/}{#2.pdf_tex}
}

\newcommand*{\ml}[2]{\multicolumn{1}{#1}{#2}}

\pdfsuppresswarningpagegroup=1


% My name
\author{Jared Anderson}
\begin{document}
		
\title{
	Answering Questions With Government Transparency Data\\
	\large An Exploration of private sector jobs
}

\maketitle
    
    

    
If I decided to get a ``real'' job, and stop being an IT contractor,
working in the public sector is an option. I know they get decent
benefits, but there are a lot of factors to weigh. This is an
exploration of government jobs and salaries pulled from Oregon's
government transparancy website.

\hypertarget{some-assumptions-and-caveats}{%
\subsection{Some Assumptions and
Caveats}\label{some-assumptions-and-caveats}}

\begin{enumerate} \item This data is real life and economic in nature, which is to say, messy
	and somewhat unnatural

  \begin{itemize}
  \item
    I will be using an $\alpha$ of 0.05–this isn't physics. As this is essentially
	economic data, this seems appropriate.
  \item
	  Confidence intervals will be set to 90\% in keeping with our generosity in terms of
	  precision.
\item
	When calculating certain statistics, I will be sampling from the data. This is to
	allow certain statistics to be computed using the tools I have learned. Other times
	this is done to estimate a simpler distribution than the entire population's real
	distribution. Any time this is done, it will be explicitly stated.
  \end{itemize}
\item
	There is a large amount of data, (approximately 27\unit{\mega\byte} of text data). As
	such, I will be making extensive use of R to calculate certain values and statistics.

  \begin{itemize}
	\item
		I will be displaying only partial and/or relevant results in tables.
	\item
		Code for the calculations will be made available at \todo{github link}

  \end{itemize}
\item
  This is for class. I'm trying to show I learned anything, sometimes
  the data makes that hard to do:

  \begin{itemize}
  \item
	  There will be times where in order to apply a particular statistical test, I will
	  make the assumption that the data is normally distributed. This may not always be
	  the case. Whenever this assumption is made, I will say so explicitly.
	\item
		When feasible, the first time a statistical test is encountered I will be showing
		how the result of a statistic is calculated. Other statistics will be summarized.
		Again see the code listing to see how a particular calculation was done.
  \end{itemize}
\end{enumerate}

    

\begin{tabular}{r|llllllll}
  & fiscal.year & agency & classification & salary.annual & full.part.time & service.type & agency.1 & gen\_class\\
  & <int> & <chr> & <chr> & <int> & <chr> & <chr> & <int> & <chr>\\
\hline
	12 & 2015 & CORRECTIONS, DEPT OF    & DENTIST          & 176352 & JOB SHARE & REPRESENTED       & 29100 & DENTIST       \\
	13 & 2015 & CORRECTIONS, DEPT OF    & DENTIST          & 176352 & PART TIME & REPRESENTED       & 29100 & DENTIST       \\
	14 & 2015 & CORRECTIONS, DEPT OF    & WELDER 2         &  70494 & FULL TIME & REPRESENTED       & 29100 & WELDER        \\
	15 & 2015 & CORRECTIONS, DEPT OF    & WELDER 2         &  70806 & FULL TIME & REPRESENTED       & 29100 & WELDER        \\
	16 & 2015 & EDUCATION, DEPT OF      & CUSTODIAN        &  31632 & PART TIME & REPRESENTED       & 58100 & CUSTODIAN     \\
	17 & 2015 & EDUCATION, DEPT OF      & CUSTODIAN        &  31632 & PART TIME & REPRESENTED       & 58100 & CUSTODIAN     \\
	18 & 2015 & EDUCATION, DEPT OF      & CUSTODIAN        &  31632 & PART TIME & REPRESENTED       & 58100 & CUSTODIAN     \\
	19 & 2015 & YOUTH AUTHORITY, OREGON & FISCAL ANALYST 2 &  69624 & FULL TIME & REPRESENTED       & 41500 & FISCAL ANALYST\\
	20 & 2015 & YOUTH AUTHORITY, OREGON & FISCAL ANALYST 3 &  62772 & FULL TIME & EXECUTIVE SERVICE & 41500 & FISCAL ANALYST\\
\end{tabular}

	\hypertarget{getting-my-feet-wet-creeping-on-people-i-met-once}{ 
		\section{An Exercise In Conditional Probability}
		\label{getting-my-feet-wet-creeping-on-people-i-met-once}}

\hypertarget{which-dept.-did-that-woman-i-met-work-at}{%
\subsection{Which Dept. Did That Woman I Met Work
At?}\label{which-dept.-did-that-woman-i-met-work-at}}

I met a woman at a python conference in early 2020. She said she worked
or had worked for the government as an analyst---I can't remember which.
We talked about what we did for a while, and I asked how well her job
payed; she said that she made ``around 65 grand'' per year.

To do this analysis, we're gonna make some assumptions and pretend some
stuff.

\begin{itemize}
\item
  analyst pay in each dept is normally distributed (it probably isn't
  exactly, but a look at some graphs look roughly normal, good enough
  for me)
\item
  Pretend I \textbf{don't} have access to the full population dataset
\end{itemize}

We're gonna pretend that I went to each department and asked 10
analyists what they made. Also I'll pretend that I did this over the
period from 2015-2020 (all samples randomly and indepently chosen).
We'll use that to estimate the distribution for each department.

    \hypertarget{step-1-filter-the-data}{%
\subsection{Step 1: Filter the data}\label{step-1-filter-the-data}}

    

    \hypertarget{step-2-see-if-theres-even-a-difference-between-the-average-salaries-in-each-department}{%
\subsection{Step 2: See If There's Even A Difference Between The Average
Salaries In Each
Department}\label{step-2-see-if-theres-even-a-difference-between-the-average-salaries-in-each-department}}

To do this, we're going to conduct an anova test. For this test we are
going to assume: 1. Each departments salary distribution is normal (same
as we said above) 2. Each departments standard deviation is the same
(Not something we're really confident about)

The groups for the anova test are going to be the departments. Here are
our hypotheses:

H0: each department pays their analysts the same on average.

HA: each department does \textbf{not} pay their analysts the same on
average.

    

    A anova: 2 × 5
\begin{tabular}{r|lllll}
  & Df & Sum Sq & Mean Sq & F value & Pr(>F)\\
  & <int> & <dbl> & <dbl> & <dbl> & <dbl>\\
\hline
	dept &  55 &  79328877331 & 1442343224 & 6.126613 & 2.759689e-30\\
	Residuals & 504 & 118653001472 &  235422622 &       NA &           NA\\
\end{tabular}


    
    

    \begin{center}
    \adjustimage{max size={0.9\linewidth}{0.9\paperheight}}{Project_files/Project_10_0.png}
    \end{center}
    { \hspace*{\fill} \\}
    
    It's pretty clear looking at the graph that there is quite a lot of
variance in the mean pay among the departments, With 2 particular
outliers in the middle.

\hypertarget{step-3-in-comes-the-bayes}{%
\subsection{Step 3: In Comes The
Bayes}\label{step-3-in-comes-the-bayes}}

    

\begin{tabular}{r|llll}
  & agency & avg & sd & p\_sal\_g\_dpt\\
\hline
	1 & ADMINISTRATIVE SRVCS, DEPT OF & 77948.4 & 16785.40 & 0.07053486\\
	2 & AGRICULTURE, DEPT OF          & 71245.2 & 14785.26 & 0.09847170\\
	3 & AVIATION, DEPARTMENT OF       & 64823.1 & 12195.19 & 0.13025460\\
	4 & BLIND, COMMISSION FOR THE     & 70572.0 & 13809.25 & 0.10621267\\
	5 & CHIEF EDUCATION OFFICE        & 69051.6 & 11633.74 & 0.12853961\\
	6 & COMM COLL/WRKFRCE DEV OFFICE  & 74422.3 & 12343.27 & 0.09642987\\
\end{tabular}


    
    

All conditional probs sum to 1:  TRUE

\begin{tabular}{r|ll}
  & agency & p\_dpt\_g\_sal\\
  & <chr> & <dbl>\\
\hline
	1 & ADMINISTRATIVE SRVCS, DEPT OF & 0.03561180\\
	8 & CONSUMER AND BUS SRVCS, DEPT  & 0.03643026\\
	46 & PUBLIC EMPS RETIREMENT SYSTEM & 0.06175604\\
	52 & TRANSPORTATION, DEPT OF       & 0.10322827\\
	39 & OREGON HEALTH AUTHORITY       & 0.18219138\\
	21 & HUMAN SERVICES, DEPARTMENT OF & 0.21858462\\
\end{tabular}


    
    

     "Hand Calculations and R agree!"
The Corrolation coeficient "r" for these two variables is  0.9829046
    
    \begin{center}
    \adjustimage{max size={0.9\linewidth}{0.9\paperheight}}{Project_files/Project_14_1.png}
    \end{center}
    { \hspace*{\fill} \\}
    
    Yep, an almost perfect positive corrolation. We could have saved a ton
of work if we had just looked at the porportions of total analysts. A
little depressing.

    \hypertarget{now-its-all-about-me}{%
\section{Now It's All About ME}\label{now-its-all-about-me}}

Now that I've become a little more familiar with the data, it's time to
do some statistics that help answer my more pressing concerns. Is it
worth it for me to try to get into the public sector?

    \hypertarget{will-the-public-sector-even-be-hiring}{%
\section{Will The Public Sector Even Be
Hiring?}\label{will-the-public-sector-even-be-hiring}}

\hypertarget{overall-year-over-year-job-growth}{%
\subsubsection{Overall Year-Over-Year Job
Growth}\label{overall-year-over-year-job-growth}}

First things first, will there even be enough growth in the govenment
for me to get hired?

    

Job Growth Estimate:  1013.798


90\% confidence interval for job growth: [ 295.7503 1731.845 ]


Expected new jobs in 2023:  -2351


P-value:  0.033577805792559

    \begin{center}
    \adjustimage{max size={0.9\linewidth}{0.9\paperheight}}{Project_files/Project_18_1.png}
    \end{center}
    { \hspace*{\fill} \\}
    
    \begin{itemize}
\item
  It looks like we may expect OR to \emph{lose} about 2300 jobs next
  year. This is likely because 2023 seems to be an outlier in growth.
  Wheter this will be the case that 2023 will be a regression to the
  mean, or that 2023 will be the start of a new boom in jobs (something
  that will make us increase our growth predictions) is anyone's guess.
\item
  We are 95\% certain that they will add on between 295 and 1731
  additional jobs on average year-over-year. Our predictive precision
  here is not that great.
\item
  our p-value indicates that with our alpha of 0.05, and a p-value of
  0.0336, we can conclude that on average they are indeed growing their
  workforce.
\item
  Growth is growth, but maybe I should hold off on applying for a job
  next year.
\end{itemize}

    \hypertarget{which-departments-are-experiencing-positive-average-growth}{%
\section{Which Departments Are Experiencing Positive Average
Growth}\label{which-departments-are-experiencing-positive-average-growth}}

    

\begin{tabular}{r|ll}
  & Dept & Growth\\
  & <chr> & <dbl>\\
\hline
	9 & HOUSING \& COMM SRVCS, DEPT OF &  19.00000\\
	1 & ADMINISTRATIVE SRVCS, DEPT OF &  19.75000\\
	18 & POLICE, OREGON STATE          &  19.90476\\
	4 & EDUCATION, DEPT OF            &  26.94048\\
	5 & EMPLOYMENT DEPT               & 157.70238\\
	10 & HUMAN SERVICES, DEPARTMENT OF & 429.65476\\
\end{tabular}


    
    

    \begin{center}
    \adjustimage{max size={0.9\linewidth}{0.9\paperheight}}{Project_files/Project_22_0.png}
    \end{center}
    { \hspace*{\fill} \\}
    
    Growth seems to be heavily concentrated almost exponentially in certain
departments

\hypertarget{looking-at-jobs-im-willing-and-qualified-to-work}{%
\section{Looking At Jobs I'm Willing And Qualified To
Work}\label{looking-at-jobs-im-willing-and-qualified-to-work}}

    

    

Rows of interesting data : 6068

\begin{tabular}{r|llllllll}
  & fiscal.year & agency & classification & salary.annual & full.part.time & service.type & agency.1 & gen\_class\\
  & <int> & <chr> & <chr> & <int> & <chr> & <chr> & <int> & <chr>\\
\hline
	300 & 2015 & ADMINISTRATIVE SRVCS, DEPT OF & ACCOUNTING TECH 2 & 33072 & FULL TIME & REPRESENTED & 10700 & ACCOUNTING TECH\\
	301 & 2015 & ADMINISTRATIVE SRVCS, DEPT OF & ACCOUNTING TECH 2 & 34476 & FULL TIME & REPRESENTED & 10700 & ACCOUNTING TECH\\
	302 & 2015 & ADMINISTRATIVE SRVCS, DEPT OF & ACCOUNTING TECH 2 & 41454 & FULL TIME & REPRESENTED & 10700 & ACCOUNTING TECH\\
	303 & 2015 & ADMINISTRATIVE SRVCS, DEPT OF & ACCOUNTING TECH 2 & 43284 & FULL TIME & REPRESENTED & 10700 & ACCOUNTING TECH\\
	304 & 2015 & ADMINISTRATIVE SRVCS, DEPT OF & ACCOUNTING TECH 2 & 45448 & FULL TIME & REPRESENTED & 10700 & ACCOUNTING TECH\\
	305 & 2015 & ADMINISTRATIVE SRVCS, DEPT OF & ACCOUNTING TECH 3 & 34476 & FULL TIME & REPRESENTED & 10700 & ACCOUNTING TECH\\
	306 & 2015 & ADMINISTRATIVE SRVCS, DEPT OF & ACCOUNTING TECH 3 & 34476 & FULL TIME & REPRESENTED & 10700 & ACCOUNTING TECH\\
	307 & 2015 & ADMINISTRATIVE SRVCS, DEPT OF & ACCOUNTING TECH 3 & 37668 & FULL TIME & REPRESENTED & 10700 & ACCOUNTING TECH\\
	308 & 2015 & ADMINISTRATIVE SRVCS, DEPT OF & ACCOUNTING TECH 3 & 37668 & FULL TIME & REPRESENTED & 10700 & ACCOUNTING TECH\\
\end{tabular}


    
    \hypertarget{how-many-new-positions-that-i-might-be-interested-in-do-i-expect-to-be-added-next-year}{%
\section{How Many New Positions That I Might Be Interested In Do I
Expect To Be Added Next
Year?}\label{how-many-new-positions-that-i-might-be-interested-in-do-i-expect-to-be-added-next-year}}

    

I estimate that there will be -2 new jobs I can apply to next year.
With a 90\% confidence interval of the average yearly growh between [ 4.734924
16.3127 ]

    \begin{center}
    \adjustimage{max size={0.9\linewidth}{0.9\paperheight}}{Project_files/Project_26_1.png}
    \end{center}
    { \hspace*{\fill} \\}
    
    

\begin{tabular}{r|llll}
  & job & new.jobs.2023 & average.growth.rate & p.val\\
  & <chr> & <dbl> & <dbl> & <dbl>\\
\hline
	1 & CUSTODIAN        &  -1.75000 & 1.583333 & 0.02628355\\
	2 & RESEARCH ANALYST & -18.03571 & 5.380952 & 0.03370504\\
\end{tabular}


    
    It looks like I'm confident that they'll be looking for custodians and
research analysts at some point. But because 2022 seems to be such an
outlier, my model expects some sort of regression-to-the-mean. So
according to the data and my statistical models, I shouldn't be planning
to get a job next year.

\hypertarget{which-jobs-that-i-could-do-have-the-most-fair-pay}{%
\section{Which Jobs That I Could Do Have The Most Fair
Pay?}\label{which-jobs-that-i-could-do-have-the-most-fair-pay}}

looking at the data, there are many jobs descriptions that are followed
by numbers. I take this to mean that you can be promoted from one level
to another. If I did one of these jobs (one's I'd be willing to do),
it's important to me that I'm working with people that are being paid
fairly---it makes for a less hostile work enviornment. In this case,
I'll settle for the qualifyer that the salaries in that job are normally
distributed.

    

    \begin{center}
    \adjustimage{max size={0.9\linewidth}{0.9\paperheight}}{Project_files/Project_29_0.png}
    \end{center}
    { \hspace*{\fill} \\}
    
    \begin{center}
    \adjustimage{max size={0.9\linewidth}{0.9\paperheight}}{Project_files/Project_29_1.png}
    \end{center}
    { \hspace*{\fill} \\}
    
    \begin{center}
    \adjustimage{max size={0.9\linewidth}{0.9\paperheight}}{Project_files/Project_29_2.png}
    \end{center}
    { \hspace*{\fill} \\}
    
    \begin{center}
    \adjustimage{max size={0.9\linewidth}{0.9\paperheight}}{Project_files/Project_29_3.png}
    \end{center}
    { \hspace*{\fill} \\}
    
    \begin{center}
    \adjustimage{max size={0.9\linewidth}{0.9\paperheight}}{Project_files/Project_29_4.png}
    \end{center}
    { \hspace*{\fill} \\}
    
    \begin{center}
    \adjustimage{max size={0.9\linewidth}{0.9\paperheight}}{Project_files/Project_29_5.png}
    \end{center}
    { \hspace*{\fill} \\}
    
    \begin{center}
    \adjustimage{max size={0.9\linewidth}{0.9\paperheight}}{Project_files/Project_29_6.png}
    \end{center}
    { \hspace*{\fill} \\}
    
    \begin{center}
    \adjustimage{max size={0.9\linewidth}{0.9\paperheight}}{Project_files/Project_29_7.png}
    \end{center}
    { \hspace*{\fill} \\}
    
    \begin{center}
    \adjustimage{max size={0.9\linewidth}{0.9\paperheight}}{Project_files/Project_29_8.png}
    \end{center}
    { \hspace*{\fill} \\}
    
    \begin{center}
    \adjustimage{max size={0.9\linewidth}{0.9\paperheight}}{Project_files/Project_29_9.png}
    \end{center}
    { \hspace*{\fill} \\}
    
    \begin{center}
    \adjustimage{max size={0.9\linewidth}{0.9\paperheight}}{Project_files/Project_29_10.png}
    \end{center}
    { \hspace*{\fill} \\}
    
    \begin{center}
    \adjustimage{max size={0.9\linewidth}{0.9\paperheight}}{Project_files/Project_29_11.png}
    \end{center}
    { \hspace*{\fill} \\}
    
    

\begin{tabular}{r|lS}
	& \ml{l}{job} & \ml{l}{shapiro.p}\\
\hline
	4 & RESEARCH ANALYST                             & 0.80949281\\
	1 & ACCOUNTING TECH                              & 0.64147751\\
	3 & ASSOCIATE IN GEOLOGY                         & 0.46369124\\
	6 & TRUCK DRIVER                                 & 0.34486630\\
	2 & ARCHIVIST                                    & 0.26270285\\
	5 & TRANSPORTATION TELECOMMUNICATIONS SPECIALIST & 0.00694868\\
\end{tabular}


    
    Well, it looks like \textbf{most} of the positions that I'm interested
have strong evidence in favor of normality. Under my definition of
egalitarianism, it looks like I'd be happy at any of these jobs except
as a transportation telecom specialist.

    % Add a bibliography block to the postdoc
    
    
\end{document}
