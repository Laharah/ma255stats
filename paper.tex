% Some basic packages
\documentclass[letterpaper]{article}
\usepackage[utf8]{inputenc}
\usepackage[T1]{fontenc}
\usepackage{textcomp}
\usepackage[english]{babel}
\usepackage{url}
\usepackage{graphicx}
\usepackage{float}
\usepackage{booktabs}
\usepackage{enumitem}
\usepackage{adjustbox}
\usepackage{hyperref}
\usepackage[binary-units]{siunitx}



%margins
\usepackage[margin=0.75in]{geometry}

% Don't indent paragraphs, leave some space between them
\usepackage{parskip}

% Hide page number when page is empty
\usepackage{emptypage}
\usepackage{subcaption}
\usepackage{multicol}
\usepackage{xcolor}

% Other font I sometimes use.
% \usepackage{cmbright}

% Math stuff
\usepackage{amsmath, amsfonts, mathtools, amsthm, amssymb}
% Fancy script capitals
\usepackage{mathrsfs}
\usepackage{cancel}
% Bold math
\usepackage{bm}
% Some shortcuts
\newcommand\N{\ensuremath{\mathbb{N}}}
\newcommand\R{\ensuremath{\mathbb{R}}}
\newcommand\Z{\ensuremath{\mathbb{Z}}}
\renewcommand\O{\ensuremath{\emptyset}}
\newcommand\Q{\ensuremath{\mathbb{Q}}}
\newcommand\C{\ensuremath{\mathbb{C}}}

% Easily typeset systems of equations (French package)
\usepackage{systeme}

% Put x \to \infty below \lim
\let\svlim\lim\def\lim{\svlim\limits}

%Make implies and impliedby shorter
\let\implies\Rightarrow
\let\impliedby\Leftarrow
\let\iff\Leftrightarrow
\let\epsilon\varepsilon

% Add \contra symbol to denote contradiction
\usepackage{stmaryrd} % for \lightning
\newcommand\contra{\scalebox{1.5}{$\lightning$}}

% \let\phi\varphi

% Command for short corrections
% Usage: 1+1=\correct{3}{2}

\definecolor{correct}{HTML}{009900}
\newcommand\correct[2]{\ensuremath{\:}{\color{red}{#1}}\ensuremath{\to }{\color{correct}{#2}}\ensuremath{\:}}
\newcommand\green[1]{{\color{correct}{#1}}}

% horizontal rule
\newcommand\hr{
    \noindent\rule[0.5ex]{\linewidth}{0.5pt}
}

% hide parts
\newcommand\hide[1]{}

% si unitx
\usepackage{siunitx}
\sisetup{locale = US}

% Environments
\makeatother
% For box around Definition, Theorem, \ldots
\usepackage{mdframed}
\mdfsetup{skipabove=1em,skipbelow=0em}
\theoremstyle{definition}
\newmdtheoremenv[nobreak=true]{lemma}{Lemma}
\newmdtheoremenv[nobreak=true]{postulate}{Postulate}
\newmdtheoremenv{conclusion}{Conclusion}
\newmdtheoremenv{assume}{Assume}
\newtheorem*{observe}{Observe}
\newtheorem*{comment}{Comment}
\newtheorem*{practice}{Practice}
\newtheorem*{problem}{Problem}
\newtheorem*{terminology}{Terminology}
\newtheorem*{application}{Application}
\newtheorem*{question}{Question}

\newmdtheoremenv[nobreak=true]{definition}{Definition}
\newtheorem*{eg}{Example}
\newtheorem*{notation}{Notation}
\newtheorem*{previouslyseen}{As previously seen}
\newtheorem*{remark}{Remark}
\newtheorem*{note}{Note}
\newtheorem*{property}{Property}
\newtheorem*{intuition}{Intuition}
\newmdtheoremenv[nobreak=true]{prop}{Proposition}
\newmdtheoremenv[nobreak=true]{theorem}{Theorem}
\newmdtheoremenv[nobreak=true]{corollary}{Corollary}

% End example and intermezzo environments with a small diamond (just like proof
% environments end with a small square)
\usepackage{etoolbox}
\AtEndEnvironment{vb}{\null\hfill$\diamond$}%
\AtEndEnvironment{intermezzo}{\null\hfill$\diamond$}%
% \AtEndEnvironment{opmerking}{\null\hfill$\diamond$}%

% Fix some spacing
% http://tex.stackexchange.com/questions/22119/how-can-i-change-the-spacing-before-theorems-with-amsthm
\makeatletter
\def\thm@space@setup{%
  \thm@preskip=\parskip \thm@postskip=0pt
}


% Exercise 
% Usage:
% \exercise{5}
% \subexercise{1}
% \subexercise{2}
% \subexercise{3}
% gives
% Exercise 5
%   Exercise 5.1
%   Exercise 5.2
%   Exercise 5.3
\newcommand{\exercise}[1]{%
    \def\@exercise{#1}%
    \subsection*{Exercise #1}
}

\newcommand{\subexercise}[1]{%
    \subsubsection*{Exercise \@exercise.#1}
}


% \lecture starts a new lecture (les in dutch)
%
% Usage:
% \lecture{1}{di 12 feb 2019 16:00}{Inleiding}
%
% This adds a section heading with the number / title of the lecture and a
% margin paragraph with the date.

% I use \dateparts here to hide the year (2019). This way, I can easily parse
% the date of each lecture unambiguously while still having a human-friendly
% short format printed to the pdf.

\usepackage{xifthen}
\def\testdateparts#1{\dateparts#1\relax}
\def\dateparts#1 #2 #3 #4 #5\relax{
    \marginpar{\small\textsf{\mbox{#1 #2 #3 #5}}}
}

% \def\@lecture{}%
% \newcommand{\lecture}[3]{
%     \ifthenelse{\isempty{#3}}{%
%         \def\@lecture{Lecture #1}%
%     }{%
%         \def\@lecture{Lecture #1: #3}%
%     }%
%     \subsection*{\@lecture}
%     \marginpar{\small\textsf{\mbox{#2}}}
% }



% These are the fancy headers
\usepackage{fancyhdr}
\pagestyle{fancy}

% LE: left even
% RO: right odd
% CE, CO: center even, center odd
% My name for when I print my lecture notes to use for an open book exam.
% \fancyhead[LE,RO]{Jared Anderson}

% \fancyhead[RO,LE]{\@lecture} % Right odd,  Left even
\fancyhead[RE,LO]{}          % Right even, Left odd

\fancyfoot[RO,LE]{\thepage}  % Right odd,  Left even
\fancyfoot[RE,LO]{}          % Right even, Left odd
\fancyfoot[C]{\leftmark}     % Center

\makeatother




% Todonotes and inline notes in fancy boxes
\usepackage{todonotes}
\usepackage{tcolorbox}

% Make boxes breakable
\tcbuselibrary{breakable}

% Verbetering is correction in Dutch
% Usage: 
% \begin{verbetering}
%     Lorem ipsum dolor sit amet, consetetur sadipscing elitr, sed diam nonumy eirmod
%     tempor invidunt ut labore et dolore magna aliquyam erat, sed diam voluptua. At
%     vero eos et accusam et justo duo dolores et ea rebum. Stet clita kasd gubergren,
%     no sea takimata sanctus est Lorem ipsum dolor sit amet.
% \end{verbetering}
\newenvironment{correction}{\begin{tcolorbox}[
    arc=0mm,
    colback=white,
    colframe=green!60!black,
    title=Opmerking,
    fonttitle=\sffamily,
    breakable
]}{\end{tcolorbox}}

% Noot is note in Dutch. Same as 'verbetering' but color of box is different
\newenvironment{aside}[1]{\begin{tcolorbox}[
    arc=0mm,
    colback=white,
    colframe=white!60!black,
    title=#1,
    fonttitle=\sffamily,
    breakable
]}{\end{tcolorbox}}




% Figure support 
\usepackage{import}
\usepackage{xifthen}
% \pdfminorversion=7
\usepackage{pdfpages}
\usepackage{transparent}
\newcommand{\incfig}[2][1]{%
    \def\svgwidth{#1\columnwidth}
    \import{./figures/}{#2.pdf_tex}
}

\newcommand*{\ml}[2]{\multicolumn{1}{#1}{#2}}

\pdfsuppresswarningpagegroup=1


% My name
\author{Jared Anderson}
\begin{document}
		
\title{
	Answering Questions With Government Transparency Data\\
	\large An Exploration of private sector jobs
}

\maketitle
    
    
My project is largely an exploration of Employee data for all government employees in
Oregon from 2015 to 2022. (Data found at
https://www.oregon.gov/transparency/Pages/State-Salaries.aspx)

There are some questions that I believe I could answer using
this data, some more relevant to me than others.

\hypertarget{some-assumptions-and-caveats}{%
\subsection{Some Assumptions and
Caveats}\label{some-assumptions-and-caveats}}

\begin{enumerate} \item This data is real life and economic in nature, which is to say, messy
	and somewhat unnatural

  \begin{itemize}
  \item
    I will be using an $\alpha$ of 0.05–this isn't physics. As this is essentially
	economic data, this seems appropriate.
  \item
	  Confidence intervals will be set to 90\% in keeping with our generosity in terms of
	  precision.
\item
	When calculating certain statistics, I will be sampling from the data. This is to
	allow certain statistics to be computed using the tools I have learned. Other times
	this is done to estimate a simpler distribution than the entire population's real
	distribution. Any time this is done, it will be explicitly stated.
  \end{itemize}
\item
	There is a large amount of data, (approximately 27\unit{\mega\byte} of text data). As
	such, I will be making extensive use of R to calculate certain values and statistics.

  \begin{itemize}
	\item
		I will be displaying only partial and/or relevant results in tables.
	\item
		Code for the calculations will be made available at \todo{github link}

  \end{itemize}
\item
  This is for class. I'm trying to show I learned anything, sometimes
  the data makes that hard to do:

  \begin{itemize}
  \item
	  There will be times where in order to apply a particular statistical test, I will
	  make the assumption that the data is normally distributed. This may not always be
	  the case. Whenever this assumption is made, I will say so explicitly.
	\item
		When feasible, the first time a statistical test is encountered I will be showing
		how the result of a statistic is calculated. Other statistics will be summarized.
		I will also be briefly summarizing the methods I used to filter and manipulate the
		data. Again see the code listing to see how a particular calculation was done.
  \end{itemize}
\end{enumerate}

    

\hypertarget{getting-my-feet-wet-creeping-on-people-i-met-once}{ 
	\section{An Exercise In Conditional Probability}
	\label{getting-my-feet-wet-creeping-on-people-i-met-once}}

I met a woman at a python conference in early 2020. She said she worked
or had worked for the government as an analyst-I can't remember which.
We talked about what we did for a while, and at one point I remember asking how well her job
payed; she said that she made ``around 65 grand'' per year.

Looking at the data, there are analysts that work in over 50 different departments. I'm
curious which department she was most most likely working at. This will let me get a
"feel" for working with the data, and offers a good opportunity to practice. But first:

\large{Assumptions \& Methods:}
\begin{itemize}
\item
	I will be assuming that analyst pay in each dept is normally distributed.
\item
	I will be sampling from the dataset to estimate these distributions.
\item
	I will be disregarding departments that have less than 10 analysts to sample from.
\end{itemize}

    \hypertarget{step-1-filter-the-data}{%
\subsection{Filtering And Sampling The Data}\label{step-1-filter-the-data}}

My plan for sampling is to do as if I had surveyed 10 analysts from each department
that employs analysts  and recorded their department and their annual salary. These
samples will be random both by individuals and by time in the range from 2015 to 2020. 


    \hypertarget{step-2-see-if-theres-even-a-difference-between-the-average-salaries-in-each-department}{%
\subsection{Is Sampling By Department Necessary?}\label{step-2-see-if-theres-even-a-difference-between-the-average-salaries-in-each-department}}

If every department pays it's analysts the same average wage, there may be no reason to
sample from each department individually. To test this, I'm going to conduct an
\textbf{Analysis of Variance} or ANOVA test. 

For this test I'm going to assume: 

\begin{enumerate}
	\item Each department's salary distribution is normal (as was stated above).
	\item Each department has the same standard deviation ($\sigma$) in the amount they
		pay their analysts.
\end{enumerate}

These two assumptions are required to conduct an accurate ANOVA test. The other assumption
requred for an ANOVA test is that the samples are simple random samples, which I have
ensured. I will point out that it is unlikely that each departments pay has the same
$\sigma^2$. However we can hope that they are similar enough to get some kind of useful
information out of the test.

The groups for the ANOVA test are going to be the departments and the values will be the
samples taken from each department. 

Our hypotheses are as such:

$\displaystyle{H_0:\forall d \in \Omega \quad  \mu_{d} = \mu_{\Omega}}$, that is: each department pays their analysts the same on average.

$\displaystyle{H_{A}: \exists  d \in \Omega \quad \text{where } \mu_{d} \neq
\mu_{\Omega}}$ or, \textbf{not} every department pays their analysts the same on average.

This results in the following ANOVA table:
    

\begin{table}[h]
	\centering
	\caption{ANOVA of Dept. Analyst Pay}
	\label{tab:1}
	\begin{tabular}{r|lllrr}
	& Df & Sum Sq & Mean Sq & F value & $P$-value\\
	\hline
		dept &  55 &  79328877331 & 1442343224 & 6.126613 & 2.759689e-30\\
		Residuals & 504 & 118653001472 &  235422622 &       &           \\
	\end{tabular}
\end{table}

This $P$-value is extremely small, giving us high confidence that the departments
\textbf{do not} pay their analysts the same on average. However given our assumption of
normality and equal variance, we should check our work visually:




    
    

\begin{figure}[H]
	\center
	\caption{Boxplot of Analysts Pay}
	\label{fig:1}
		\adjustimage{max size={0.5\linewidth}{0.5\paperheight}, caption={Barchart of
		Sampled Analyst Pay}, label={figure 1}}{Project_files/Project_13_0.png}
\end{figure}
    
    It's pretty clear looking at the chart that there is quite a lot of
variance in the pay among the departments, With a few particular
outliers in the middle.


\hypertarget{step-3-in-comes-the-bayes}{%
\subsection{Bayesian Analysis}\label{step-3-in-comes-the-bayes}}

Now, to calculate the probability of the woman I spoke to working in any given department,
I must calculate the \emph{conditional probability} that she works there. In other words,
We will calculate the probability that she works at a particular department \emph{given}
that she is paid a certain amount. This is written as $Pr\{\text{working in dept. }x | \text{she is paid }y\}$

The formula to find this is called Bayes formula and it is written:

\[
Pr \left\{ A|B \right\} = \frac{P(B|A)P(A)}{P(B)}
.\] 

Using these probabilities as analogs for elements of our problem, we assign them like so:

\begin{align*}
	P(A) &= \text{The probability that she works at a particular department}\\
	P(B|A) &= \text{The probability that she makes gets a certain salary \emph{assuming}
	that we know} \\ &\text{ she works at a particular department}\\
	P(B) &= \text{The \textbf{total} probability that she makes a particular salary}\\
	Pr\left\{ A|B \right\} &= \text{The probability that she works at a particular
	department \emph{given} that we know} \\ &\text{ what her salary is. This is the value we're
	looking for}
\end{align*}

Now I'll cover \emph{how} you would extract these number from the data.

\begin{itemize}
	\item 
		$P(A)$: We can calculate this by taking a department, counting the \textbf{total}
		number of employees there, and dividing by the total of \textbf{all} the
		employees.
	\item
		$P(B|A)$ : Since we've assumed that all departments pay their analysts with a
		normal distribution, we can use our samples to estimate the average and standard
		deviation of a particular department's payscale. Using this estimated normal
		distribution,  we can calculate the probabilty that she makes a certain amount of
		money (within an arbitrary but reasonable range)  assuming that she worked there.
	\item
		$P(B)$: To calculate this we must \emph{sum} $P(B|A)$ across  \textbf{all}
		departments.
\end{itemize}

Solving the above equation for every department, we can find the Departments with the
highest probability that she worked there. The only variable that I need to supply is the
range of salaries we should be testing for. I decided (again arbitrarily) that given she
said she makes about \$65,000, a good lower range is about \$63,000, any less and she
would probably have said she makes about \$60k. Likewise I set the upper bound of the
range to \$67,000, any more and I think it likely she would have told me she makes about
\$70k.

Here is a look at our most significant results.

\begin{table}[h]
	\center
	\caption{Most Likely Depts. My Acquaintance Worked }
	\label{tab:2}
	\begin{tabular}{ll}
  Department & $Pr\left\{ Dept. | sal. \right\} $\\
\hline
	 ADMINISTRATIVE SRVCS, DEPT OF & 0.03561180\\
	 CONSUMER AND BUS SRVCS, DEPT  & 0.03643026\\
	 PUBLIC EMPS RETIREMENT SYSTEM & 0.06175604\\
	 TRANSPORTATION, DEPT OF       & 0.10322827\\
	 OREGON HEALTH AUTHORITY       & 0.18219138\\
	 HUMAN SERVICES, DEPARTMENT OF & 0.21858462\\
\end{tabular}
\end{table}

looking at the table, we can conclude that It is most likely that she was working in the
Department of Human Services, the Oregon Heath Authority, or the Department of
Transportation (in that order)

As a sanity check, we can check an make sure that the sum of all these probabilities is
equal to one-which I am happy to report is the case. 

\subsection{Was This A Waste of Time?}

Having been looking at this data for a while I couldn't help but notice that the top 3
most likely departments are also the departments that employ the most analysts. 

This makes me wonder if we could have simply found the departments that employed the most
analysts and been just as well off. 

To check, I'll graph our calculated probabilities against the proportion of all analysts
each department employs and calculate the correlation coefficient.

    
\begin{figure}[H]
	\caption{How much Time Did I Waste?}
	\label{fig:2}
	\begin{center}
	\adjustimage{max size={0.5\linewidth}{0.5\paperheight}}{Project_files/Project_17_1.png}
	\end{center}
\end{figure}
    
After looking at the graph, it should not surprise you to hear that the calculated
correlation coefficient, $r=0.9829$ ; an almost perfect one-to-one relationship. We could
have saved a ton of work if we had just looked at the porportions of total analysts for
each department. It should be noted that the most likely Departments (Human Services, The
Health Authority, and Dept. of Transportation), are true outliers. The lower section of
the graph looks much more interesting. If we somehow knew she \emph{didn't} work in those
departments, we could eliminate them from our calculations and hopefully get more
interesting results.

    \hypertarget{now-its-all-about-me}{%
\section{Should I Get a Job in Government?}\label{now-its-all-about-me}}

I have been working for myself as an IT contractor for more than a decade now. While I
like the pay and the near total flexibility it affords me, there are certain charms to the
idea of having a "normal" job. Structure, a matching retirement account, and above all
else: sweet sweet healthcare that you don't have to pay for.

The pros and cons are difficult to weigh against each other-there are a lot of factors.
The next few questions will concern answering questions that could help me make such a
decision.


    \hypertarget{will-the-public-sector-even-be-hiring}{%
\subsection{Is The Oregon Public Sector Even Hiring?}\label{will-the-public-sector-even-be-hiring}}

How difficult would it be for me to get a job in government-any job? To help me answer
this question, I can analyse the data to see if Oregon is likely open more positions over
time. Also, I could use this information to calculate how many jobs I can expect them to
add this year.

To calculate this statistic, we will first gather the entire dataset (our whole
"population") and organize it by year and then count the total number of employees for
that given year.

Once that's been done (not as easy as it sounds when you don't really know R), we get a
set of values that looks like this:

\begin{table}[H]
	\centering
	\caption{Total Workers Per Year}
	\label{tab:label}
	\begin{tabular}{r|llllllll}
		Year & 2015& 2016&2017&2018&2019&2020&2021&2023\\
		Workers & 36767&37064&33219&37187&38033&39160&38740&45068
	\end{tabular}
\end{table}

Now that we have these values it's fairly trivial to get the mean and standard deviation
for both rows of data:

\begin{align*}
	\overline{x}&= 2018.5 \\
	\overline{y}&= 38154.875 \\
	s_{x} &= 2.4495 \\
	s_{y}&= 3329.0208 \\
.\end{align*}

Using these values I can now calculate the
\textbf{correlation coefficient}, that I somewhat glossed over above. The correlation
coefficient is denoted by the letter $r$, and it's equation for a given set of points is:

\[
r_{x, y} = \frac{1}{n-1}\left[ \left( \frac{x_{1}-\mu_{x}}{\sigma_{x}} \right)\left(
\frac{y_{1}-\mu_{y}}{\sigma_{y}} \right) +
\left(\frac{(x_{2}-\mu_{x})}{\sigma_{x}}\right)\left( \frac{y_{2}-\mu_{y}}{\sigma_{y}} \right) + \dots \right]
.\] 

If we plug in the values above and solve, we get $r=0.7460$, which indicates a somewhat
strong correlation between the year and the number of government employees.
Using this value we can then calculate the least squares linear regression line of the data. 

The formula of the LSLR line is

\[
\hat{y} = \beta_0 +\beta_1x
.\] 

where $\beta_1$ is the average rate of change in the number of workers per year, and
$\beta_0$ is the $y$ intercept of the data (which in our case is somewhat nonsense, as
there is no year 0 A.D, and if there was, Oregon wouldn't have existed, and
it certainly couldn't have had the millions of "negative" employees the model predicts it to
have had).

Notice that we don't have to estimate this regression line, because we have access to the
actual values from the entire population. The coefficients ($\beta_0$ and $\beta_1$ ) can
be calculated as such:

\begin{align}
\beta_{1}  &= r \frac{\sigma_{y}}{\sigma_{x}} \\
\beta_{0}  & = \mu_y-\beta_{1}\mu_x
\end{align}

Plugging in our numbers from above and solving gives us:

\[
\hat{y} = 1014x - 2008196
.\] 

This equation will allow us to graph both our points and the least squares regression line
like so:

\begin{figure}[H]
	\caption{Yearly Oregon Government Job Growth}
	\label{fig:3}
	\center
	\adjustimage{max size={0.5\linewidth}{0.5\paperheight}}{Project_files/Project_21_6.png}
\end{figure}
    

Lets see how confident we can be that we've gotten the slope of the regression line by
calculating a \textbf{confidence interval}. We'll need two things:

\begin{enumerate}
	\item The standard error of $\beta_1$: \[
	\sqrt{ \frac{1}{n-2} \sum_{i=1}^{n}(y_{i}-\hat{y}_{i})^{2} }
	.\] 
\item and the quantile of the student's-$t$ distribution with degrees of freedom $d$ over
	the interval $[\alpha / 2, 1-\alpha / 2]$:
	\[
	t_{d, \alpha / 2}
	.\] 
\end{enumerate}
    
The standard error has a term $s_e$ in it that hasn't been calculated yet. It's formula
is:
\[
s_{e} = s_{y|x} = \sqrt{ \frac{1}{n-2} \sum_{i=1}^{n}(y_{i}-\hat{y}_{i})^{2} }
.\] 

I leave the calculation of $SE_{\beta_1}$ as an exercise for the reader.

Plugging in our values and getting the $t$-distribution quantiles from a computer or a
table we get

\begin{align}
	SE_{\beta_1}&= 369.5217 \\
	t_{7, 0.25} &= 1.9432 \\
\end{align}

Now we can finally calculate our confidence interval:

\begin{align*}
\text{C.I}&=  \beta_1 \pm t_{n-2, \alpha / 2} \times SE_{\beta_1}\\
	  &= [295.7503, 1731.845] \\
.\end{align*}

which to be frank is not great for a 90\% confidence interval. It seems we have a bit of a
precision problem. That said, let's test our hypothesis that the number of employees is
rising year-over-year \emph{at all}. To be more formal let's state our null and
alternative hypotheses.

\begin{align*}
	H_0&: \beta_1 \le   0 \quad \text{or there is a negative or no difference in employees each year}\\
	H_A&: \beta_1 > 0 \quad \text{or there is a positive correlation between the year and
	the number of employees.}
\end{align*}

To calculate the $p$-value we'll calculate a $t$-stat first.

\begin{align}
	t \text{-stat} &= \frac{\beta_1 - m_0}{SE_{\beta_1}} \\
	&= 2.7435 \\
\end{align}
And the last thing we need is to use the $t$-distribution to get our $p$-value:

\begin{align*}
	P\text{-Value} &= Pr \left\{ t_{n-2} > t \text{-stat} | H_0 \right\}  \\
	&= 0.0168 \\
.\end{align*}

As I said in the beginning, we've given ourselves a generous $\alpha$ of 0.05. This
$P$-value gets us well under that line, which means that we can confidently
\textbf{reject} the null hypothesis.

Technically I've answered my question, I'm confident that the number of government
employees is growing. But I want to be a little more specific; if I apply for a position
this year, do I expect there to be a job there for me?

Answering this is not only straight forward, but easy. All I have to do is plug in the
desired year into my least square linear regression model:

\begin{align*}
	\hat{y} &= \lfloor \beta_0 + \beta_1(2023)\rfloor\\
	&=  -2351\\
.\end{align*}	

It seems that the answer to that question is a resounding no---my model predicts that Oregon
will lay off more than 2000 people next year. It seems like the reason for this result is
that 2023 was a noticeable positive outlier in terms of number of employees. My model
expects there to be some kind of regression to the mean. But I suppose it's possible that
there is a factor or variable that can't be found in my data, and 2022 marks the beginning
of a new growth period. 




    

    
    \hypertarget{how-many-new-positions-that-i-might-be-interested-in-do-i-expect-to-be-added-next-year}{%
\subsection{How Many New Positions That I Might Be Interested In Do I
Expect To Be Added Next
Year?}\label{how-many-new-positions-that-i-might-be-interested-in-do-i-expect-to-be-added-next-year}}

First I will filter out jobs that I'm not particularly interested in doing. This leaves
me with a dataset of 6068 jobs over the last eight years. Of these jobs, how many do I
expect will have openings this year?

To answer this question we won't really be doing anything new. We'll be doing more or less
the same as we did in section 2.1. First we'll group together each remaining job by year
and calculate a least square regression line. A quick summary of the data and variables is
listed here:
\begin{table}[H]
	\centering
	\caption{Candidate Jobs Per Year}
	\label{tab:label}
	\begin{tabular}{r|llllllll}
		Year & 2015& 2016&2017&2018&2019&2020&2021&2023\\
		Jobs & 741& 740& 707& 739& 781&773& 789&798
	\end{tabular}
\end{table}

\begin{align*}
	r &= 0.8218\\
	\beta_1&= 10.524 \\
	\beta_0	&= -20483.810 \\
	SE_{\beta_1} &= 2.979 \\
	t \text{-value} &= 3.533 \\
.\end{align*}	

Frankly I'm more interested in what I can roughly expect than I am about the surety that
my trend line is positive, so I'll calculate a 90\% confidence interval. Honestly that's
more confident than I've been about almost anything in real life lately which makes it good
enough for me:

\begin{align*}
	\text{C.I.} &= \beta_1 \pm t_{6, 0.05 / 2} \times SE_{\beta_1} \\
				&= [4.7349, 16.313] \\
.\end{align*}

That gives me some hope, it certainly seems likely that there is slow but steady growth in
my candidate jobs. Let's take a look at the graph:
\begin{figure}{H}
	\center
	\caption{Candidate Jobs Per Year}
	\label{fig:4}
	\adjustimage{max size={0.5\linewidth}{0.5\paperheight}}{Project_files/Project_29_4.png}
\end{figure}

Now to see how many openings total I should expect this year:

\begin{align*}
	\hat{y} &= \lfloor 10.524(2023)-20483.810 \rfloor \\
	&=  -2\\
.\end{align*}


Once again, I am likely out of luck if I'm looking to switch into a government job in the immediate future.

\subsection{Which of The Specific Jobs I'm Willing To Do Are Growing?}

While I may be fairly confident that there is \emph{overall} growth in my candidate jobs,
which specifically are those? If they're just jobs that I can tolerate, without much
window for growth, I may not be terribly interested.

To answer this question, I'm going to separate the jobs not only by year, but also by
job type. This will give us 23 jobs with 8 years of records each.

I'll then take each job type, and calculate a least square linear regression on the number
of employees in that job type based on the year.  I'll filter this data \emph{again} by
ignoring any regressions that have negative growth but also any jobs where I'm not
confident enough that my calculated growth isn't a mirage (where the $P$-value is too
high.)

Though I already expect the answer, I'll also calculate how many openings in those
jobs I can expect this year.

All of which leaves me with this:

\begin{table}[htpb]
	\centering
	\caption{Most Likely Career I Can Get}
	\label{tab:label}
	\begin{tabular}{l|lll}
	  Career  & Average Growth Rate & $P$-value & Estimated Openings in 2023\\
	\hline
		CUSTODIAN         & 1.583333 & 0.02628355&  -1\\
		RESEARCH ANALYST  & 5.380952 & 0.03370504& -18\\
	
	\end{tabular}
\end{table}

Whelp, I'm glad that at least \emph{something} has edged out custodian. It seems to have
done so fairly well too in terms of growth. My model predicts much more research analysts
to be let go next year. This is likely a function of the fact that there are many more
researchers in the dataset, and that the number of custodians you need only goes up when
you expand offices.

\hypertarget{which-jobs-that-i-could-do-have-the-most-fair-pay}{%
\subsection{Which Jobs That I Could Do Have The Most Fair
Pay?}\label{which-jobs-that-i-could-do-have-the-most-fair-pay}}

looking at the data, there are many jobs descriptions that are followed
by numbers. I take this to mean that you can be promoted from one level
to another. If I took one of these jobs (again these are ones I'd be willing to actually do),
it's important to me that I'm working with people that are being paid
fairly---it makes for a less hostile work environment. In this case,
I'll settle for the qualifier that the salaries in that job are normally
distributed.


It's important that I be able to quantify a distribution's "normalness" so that I can rank
one career against another. To do this I'm going to be using a Shapiro-Wilk test. Because
the Shapiro-Wilk test tends to overfit with any sample size of about 60 or so, I'm going
to sample from the population of each career. I'll take 30 simple random samples from each
one to use in the test.

Before I do that though, I want to get an intuitive sense for how normal these
distributions are. To do that I'm going to graph histograms and Q-Q plots for a few
randomly selected careers.

A Q-Q plot is made by placing your observations in rank order ($y_1\le y_2\le \dots \le
y_n $). We then shift these points slightly off center. Not by much, just enough so that
we don't get strange values at the edges. We then scale these values by a factor of $1 /
n$. In more concrete terms, if $j$ is the $j$th order of the numbers go through the
function

\[
f(j) = \frac{j-1 / 2}{n}
.\] 

This pulls the numbers in so that $y_{(j)}$, roughly moves to it's quantile position so
long as the distribution was normal to start with. These numbers become the normal scores or $n$-scores. 

Once done, we can plot these $n$-scores against an actually standardized set of ordered
\emph{values}, $w_{(j)}$ :

\[
w_{(j)} = \frac{y_{(j)} - \overline{y}}{s_y} \approx z_{(j)}
.\] 	

All of this allows you to plot the $w_{(j)}$ values against the "maybe" standardized
values. 

If the values were sampled from a roughly normal distribution, the graph should
approximate a straight line along the $(1,1)$ slope. 

Here are a couple of the histograms and Q-Q plots for you to compare. 


\begin{figure}[H]
		\centering
		\caption{Sample Histograms and Their QQ-Plots}
		\label{fig:label}
		\begin{tabular}{cc}
		
    \adjustimage{max
	size={0.225\linewidth}{0.225\paperheight}}{Project_files/Project_33_0.png} &
	\adjustimage{max
	size={0.225\linewidth}{0.225\paperheight}}{Project_files/Project_33_1.png}\\
    \adjustimage{max
size={0.225\linewidth}{0.225\paperheight}}{Project_files/Project_33_2.png} & 
    \adjustimage{max size={0.225\linewidth}{0.225\paperheight}}{Project_files/Project_33_3.png}

		\end{tabular}
\end{figure}

By looking at the graphs, you can see that the histograms are roughly normal (if you
squint), but the Q-Q graphs are more clearly aligned along the 45 degree axis.

Finally we can put our samples through the Shapiro-Wilk test and take a look at the
results:
\begin{table}[htpb]
	\centering
	\caption{Careers and Their Salary "Fairness"}
		\begin{tabular}{ll}
			Career & Shapiro $P$-val\\
		\hline
			 RESEARCH ANALYST                             & 0.80949281\\
			 ACCOUNTING TECH                              & 0.64147751\\
			 ASSOCIATE IN GEOLOGY                         & 0.46369124\\
			 TRUCK DRIVER                                 & 0.34486630\\
			 ARCHIVIST                                    & 0.26270285\\
			 TRANSPORTATION TELECOMMUNICATIONS SPECIALIST & 0.00694868\\
		\end{tabular}
\end{table}



    
    Well, it looks like \textbf{most} of the positions that I'm interested
have strong evidence in favor of normality. Under my definition of
egalitarianism, it looks like I'd be happy at any of these jobs except
as a transportation telecom specialist (which is okay since I don't really know what that
is.

\subsection{Conclusion}

I can't say that I've been completely swayed one way or the other as far as taking a
government job is concerned. I certainly think that it's not something I should decide
this year given my predictions from this data. I was surprised to see that most of the
careers I'm interested in have "fair" pay. 

To really help me decide, I should attempt to collect similar data for the private sector,
match up like jobs, compare their salaries, growth, and "fairness". 

    
\end{document}
